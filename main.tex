\documentclass[]{article}
\usepackage[utf8]{inputenc}
\usepackage[italian]{babel}
\usepackage{geometry}
\usepackage{changepage}
\usepackage{subcaption}
\usepackage{lipsum}
\usepackage{hyperref}
\usepackage{comment}

%opening
\title{\textbf{Guida all'installazione di ROOT in ambiente macOS}}
\author{Matteo Marchisio Caprioglio}

\begin{document}

\maketitle

\begin{abstract}
	Questa breve guida ha lo scopo di illustrare passo passo la procedura di installazione e setup del software ROOT in ambiente macOS. Si descrivono tre procedure alternative:
\begin{itemize}
    \item Una di livello base, che sfrutta l'utilizzo del packagemanager \texttt{Homebrew} per installare una distribuzione pre-compilata di ROOT (sezione \ref{}).
    \item Una di livello medio, che prevede l'utilizzo del terminale per installare una distribuzione pre-compilata di ROOT con qualche accortezza in più (sezione \ref{}).
    \item Una di livello avanzato, che illustra la procedura di build locale di ROOT da codice sorgente, scaricato tramite clonazione della repository Github ufficiale.
\end{itemize}
Si illustra anche qualche nozione basilare sull'utilizzo di comandi da shell e dell'editor di testo Vim.
\end{abstract}

\newpage

\tableofcontents
\newpage

\section{Controlli e setup preliminari}
\label{sec:preliminari}
In questa sezione si descrivono alcuni passaggi preliminari necessari all'installazione di ROOT a prescindere dalla procedura adottata, oltre ad alcuni termini adottati nel seguito della guida.

\subsection{Controlli preliminari}
\label{sec:prelim-checks}
È innanzitutto necessario installare tutte le dependencies necessarie all'installazione di ROOT \cite{root_required_dependecies}. Queste sono:
\begin{itemize}
    \item XCode: applicazione di sviluppo software per macOS, scaricabile gratuitamente dall'App Store.
    \item XCode Command Line Tools: suite di strumenti da Terminale necessari in background. Per installarli aprire l'applicazione \texttt{Terminale} e digitare il comando
    \begin{verbatim}
    xcode-select --install
    \end{verbatim}
\end{itemize}



\section{Installazione da distribuzione pre compilata}
\label{sec:pre-comp}
%Da sistemare


\section{Controlli preliminari}
\begin{itemize}
	\item Verificare la versione di macOS e il tipo di processore (per farlo cliccare sulla mela nella barra in alto, quindi \texttt{Informazioni su questo Mac})
	\item Assicurarsi di aver installato XCode, e controllarne la versione (per farlo aprire XCode, poi selezionare \texttt{XCode} nella barra in alto, quindi \texttt{About XCode})
	\item Installare i componenti aggiuntivi, i cosiddetti \texttt{Command Line Developer Tools}: per farlo aprire l'applicazione \textbf{Terminale} ed eseguire il comando 
	\begin{verbatim}
		xcode-select --install
	\end{verbatim}
	copiandolo nel terminale e premendo INVIO
\end{itemize}

\section{Download e installazione di ROOT}
\begin{itemize}
	\item Aprire il Terminale e portarsi nella cartella utente con il comando \texttt{cd}
	\item Collegarsi a \hyperref{https://root.cern/install/all_releases/}{}{}{questo link} per scegliere la versione di ROOT da installare (si consiglia di scegliere sempre la versione \textbf{Latest Stable})
	\item Tra le versioni diponibili per macOS, i cui link hanno la forma
	\begin{verbatim}
		root_v6.28.06.macos-13.5-x86_64-clang140.tar.gz
	\end{verbatim}
	scegliere quella adeguata alle versioni di macOS e XCode e al tipo di processore del proprio computer (\texttt{x86\_64} per processori Intel, \texttt{arm64} per processori M1, M2 ecc. proprietari Apple)
	\item Cliccare col tasto destro sul link che finisce con \texttt{tar.gz} e selezionare \texttt{Copia link} 
	\item Tornare al Terminale e digitare il comando
	\begin{verbatim}
		curl -LO <link_versione>
	\end{verbatim}
	in cui al posto di \texttt{<link\_versione>} basta incollare il link copiato in precedenza, poi premere INVIO
	\item Verificare che il download sia avvenuto con successo con il comando \texttt{ls}, tra i risultati dovrebbe comparire il \texttt{tar.gz} della versione scelta
	\item Estrarre il \texttt{tar.gz} digitando il comando
	\begin{verbatim}
		tar -xzvf <nome_versione>.tar.gz
	\end{verbatim}
	e premendo INVIO: l'estrazione avviene con la creazione automatica di una cartella chiamata semplicemente \texttt{root} (verificarne la presenza con il comando \texttt{ls})
\end{itemize}

\section{Setup}
Per avviare ROOT da terminale non è sufficiente scrivere semplicemente \texttt{root} e premere INVIO, perchè va "pescato" il suo funzionamento dalla cartella \texttt{root} creata in fase di installazione. I seguenti passaggi permettono di modificare i file di configurazione del Terminale, in modo da non dover rieseguire la stessa stringa di codice ogni volta che si riapre il Terminale prima di eseguire ROOT.
\begin{itemize}
	\item Aprire col Finder la cartella utente, selezionare col tasto destro la cartella \texttt{root} creata in fase di installazione e cliccare \texttt{Ottieni informazioni}: alla dicitura "Situato in" compare il percorso della cartella, copiarlo cliccandoci su col tasto destro (\texttt{Copia come percorso})
	\item Aprire il Terminale verificare il tipo di shell utilizzato, con il comando
	\begin{verbatim}
		echo $0
	\end{verbatim}
	Lo shell di default è \texttt{zsh}, ma è possibile utilizzare anche \texttt{bash} in modo equivalente (a \hyperref{https://www.howtogeek.com/444596/how-to-change-the-default-shell-to-bash-in-macos-catalina/}{}{}{questo link} è spiegata la procedura per cambiare tra uno e l'altro: ai fini dell'utilizzo di ROOT \texttt{zsh} funziona, ed è pertanto sconsigliato il cambio di shell agli utenti meno esperti)
	\item Se lo shell in uso è \texttt{zsh}, digitare nel terminale il comando
	\begin{verbatim}
		sudo vim .zshrc
	\end{verbatim}
	in questo modo si apre (o si crea se non esiste) con i privilegi da amministratore (\texttt{sudo}) il file di configurazione nascosto \texttt{.zshrc} conl'editor di testo Vim
	\item Se lo shell in uso è \texttt{bash}, in modo analogo digitare nel terminale il comando
	\begin{verbatim}
		sudo vim .bashrc
	\end{verbatim}
	oppure
	\begin{verbatim}
		sudo vim .bash_profile
	\end{verbatim}
	\item Una volta aperto con Vim il file di configurazione nascosto con la procedura appena descritta, cliccare I per portarsi nella modalità inserimento di Vim (compare la scritta INSERT al fondo dello schermo), poi portarsi con le frecce al fondo del file e incollare la stringa
	\begin{verbatim}
		source <percorso_root>/root/bin/thisroot.sh
	\end{verbatim}
	in cui al posto di \texttt{<percorso\_root>} va incollato il percorso della cartella \texttt{root}, trovato al primo passaggio di questa lista
	\item Uscire dalla modalità inserimento di Vim cliccando ESC, poi salvare il file digitando il comando \texttt{:wq} e premendo INVIO
	\newpage
	\item Se tutto ha funzionato correttamente, dopo aver chiuso e riaperto il terminale se si digita il comando \texttt{root} dovrebbe avviarsi ROOT con un output simile a
	\begin{verbatim}
		 ------------------------------------------------------------------
		| Welcome to ROOT 6.28/06                        https://root.cern |
		| (c) 1995-2023, The ROOT Team; conception: R. Brun, F. Rademakers |
		| Built for macosx64 on Nov 25 2023, 11:48:00                      |
		| From heads/latest-stable@7745d36d                                |
		| With Apple clang version 15.0.0 (clang-1500.0.40.1)              |
		| Try '.help'/'.?', '.demo', '.license', '.credits', '.quit'/'.q'  |
		 ------------------------------------------------------------------
		
		root [0]
	\end{verbatim}
\end{itemize}

\section{Eseguire macro (file in formato .C) con ROOT}
\begin{itemize}
	\item Aprire il terminale e portarsi nella cartella in cui sono salvate le macro .C che si vogliono eseguire, digitando il comando
	\begin{verbatim}
		cd <percorso_macro>
	\end{verbatim}
	in cui al posto di \texttt{<percorso\_macro>} va incollato il percorso delle macro (ottenuto con la solita procedura: Finder, clic destro sul file, Ottieni Informazioni, Situato in)
	\item Avviare ROOT col comando \texttt{root}
	\item Digitare il comando
	\begin{verbatim}
		.x <nome_file>.C
	\end{verbatim}
	se la macro che si vuole eseguire e la funzione principale al suo interno hanno lo stesso nome, altrimenti
	\begin{verbatim}
		.L <nome_file>.C
	\end{verbatim}
	seguito da
	\begin{verbatim}
		funzione_da_eseguire(eventuali argomenti)
	\end{verbatim}
	\item per uscire da ROOT digitare il comando \texttt{.q}
\end{itemize}


\section{Build locale da codice sorgente}
\label{sec:build}
\input{sections/build}


\newpage

\begin{comment}
\section{Documentazione}
\begin{itemize}
	\item \hyperref{https://root.cern}{}{}{Sito di ROOT - documentazione completa}
	\item \hyperref{https://root.cern/install/all_releases/}{}{}{Releases di ROOT}
	\item \hyperref{https://www.howtogeek.com/444596/how-to-change-the-default-shell-to-bash-in-macos-catalina/}{}{}{Come cambiare la shell di default di macOS}
	\item \hyperref{https://www.gnu.org/savannah-checkouts/gnu/bash/manual/bash.html#Controlling-the-Prompt}{}{}{Manuale comandi shell}
	\item \hyperref{https://wiki.gentoo.org/wiki/Vim/Guide}{}{}{Guida Vim}
\end{itemize}
\end{comment}

\bibliographystyle{plain}
\bibliography{biblio}



\end{document}
