Questa procedura è la più semplice delle tre proposte in questa guida, e prevede l'utilizzo del package manager \texttt{HomeBrew} per installare una distribuzione pre-compilata di ROOT (quasi) pronta all'uso.\\

\subsubsection{Installazione di HomeBrew}
HomeBrew è un package manager gratuito per macOS che permette di installare facilmente software open source di vario tipo. \\
Per installarlo, seguire una delle procedure descritte sul sito ufficiale di Homebrew \cite{brew_page} (si consiglia quella che prevede il download di un file formato .pkg direttamente da GitHub, ma anche quella che sfrutta il terminale funziona).\\

\subsubsection{Download ROOT con HomeBrew}
Una volta installato HomeBrew, aprire il Terminale ed eseguire il comando
\begin{verbatim}
brew install root
\end{verbatim}
