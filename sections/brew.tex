Questa procedura è la più semplice delle tre proposte in questa guida, e prevede l'utilizzo del package manager \texttt{HomeBrew} per installare una distribuzione pre-compilata di ROOT (quasi) pronta all'uso.\\

\subsection{Installazione di HomeBrew}
\label{brew-install}
HomeBrew è un package manager gratuito per macOS che permette di installare facilmente software open source di vario tipo. \\
Per installarlo, aprire il terminale e digitare il seguente comando:
\begin{lstlisting}
/bin/bash -c "$(curl -fsSL https://raw.githubusercontent.com/Homebrew/install/HEAD/install.sh)"
\end{lstlisting}