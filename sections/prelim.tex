In questa sezione si descrivono alcuni passaggi preliminari necessari all'installazione di ROOT a prescindere dalla procedura adottata.\\
È innanzitutto necessario installare tutte le dependencies richieste all'installazione di ROOT \cite{root_required_dependecies}. Per tutte le procedure descritte nel seguito queste sono:
\begin{itemize}
    \item XCode: applicazione di sviluppo software per macOS, scaricabile gratuitamente dall'App Store.
    \item XCode Command Line Tools: suite di strumenti da Terminale necessari in background. Per installarli aprire l'applicazione \texttt{Terminale} ed eseguire il seguente comando digitandolo e premendo INVIO:
    \begin{verbatim}
    xcode-select --install
    \end{verbatim}
\end{itemize}
È inoltre consigliato di segnarsi preventivamente varie informazioni riguardo al proprio Mac che possono essere utili nel seguito della guida:
\begin{itemize}
    \item Versione di macOS e tipologia di processore: cliccare sulla mela in alto a sinistra, quindi \texttt{Informazioni su questo Mac/About this Mac}.
    \item Versione di XCode: aprire XCode, quindi selezionare \texttt{XCode} nella barra in alto, quindi \texttt{Informazioni su XCode/About XCode}.
\end{itemize}


