%Da sistemare


\section{Controlli preliminari}
\begin{itemize}
	\item Verificare la versione di macOS e il tipo di processore (per farlo cliccare sulla mela nella barra in alto, quindi \texttt{Informazioni su questo Mac})
	\item Assicurarsi di aver installato XCode, e controllarne la versione (per farlo aprire XCode, poi selezionare \texttt{XCode} nella barra in alto, quindi \texttt{About XCode})
	\item Installare i componenti aggiuntivi, i cosiddetti \texttt{Command Line Developer Tools}: per farlo aprire l'applicazione \textbf{Terminale} ed eseguire il comando 
	\begin{verbatim}
		xcode-select --install
	\end{verbatim}
	copiandolo nel terminale e premendo INVIO
\end{itemize}

\section{Download e installazione di ROOT}
\begin{itemize}
	\item Aprire il Terminale e portarsi nella cartella utente con il comando \texttt{cd}
	\item Collegarsi a \hyperref{https://root.cern/install/all_releases/}{}{}{questo link} per scegliere la versione di ROOT da installare (si consiglia di scegliere sempre la versione \textbf{Latest Stable})
	\item Tra le versioni diponibili per macOS, i cui link hanno la forma
	\begin{verbatim}
		root_v6.28.06.macos-13.5-x86_64-clang140.tar.gz
	\end{verbatim}
	scegliere quella adeguata alle versioni di macOS e XCode e al tipo di processore del proprio computer (\texttt{x86\_64} per processori Intel, \texttt{arm64} per processori M1, M2 ecc. proprietari Apple)
	\item Cliccare col tasto destro sul link che finisce con \texttt{tar.gz} e selezionare \texttt{Copia link} 
	\item Tornare al Terminale e digitare il comando
	\begin{verbatim}
		curl -LO <link_versione>
	\end{verbatim}
	in cui al posto di \texttt{<link\_versione>} basta incollare il link copiato in precedenza, poi premere INVIO
	\item Verificare che il download sia avvenuto con successo con il comando \texttt{ls}, tra i risultati dovrebbe comparire il \texttt{tar.gz} della versione scelta
	\item Estrarre il \texttt{tar.gz} digitando il comando
	\begin{verbatim}
		tar -xzvf <nome_versione>.tar.gz
	\end{verbatim}
	e premendo INVIO: l'estrazione avviene con la creazione automatica di una cartella chiamata semplicemente \texttt{root} (verificarne la presenza con il comando \texttt{ls})
\end{itemize}

\section{Eseguire macro (file in formato .C) con ROOT}
\begin{itemize}
	\item Aprire il terminale e portarsi nella cartella in cui sono salvate le macro .C che si vogliono eseguire, digitando il comando
	\begin{verbatim}
		cd <percorso_macro>
	\end{verbatim}
	in cui al posto di \texttt{<percorso\_macro>} va incollato il percorso delle macro (ottenuto con la solita procedura: Finder, clic destro sul file, Ottieni Informazioni, Situato in)
	\item Avviare ROOT col comando \texttt{root}
	\item Digitare il comando
	\begin{verbatim}
		.x <nome_file>.C
	\end{verbatim}
	se la macro che si vuole eseguire e la funzione principale al suo interno hanno lo stesso nome, altrimenti
	\begin{verbatim}
		.L <nome_file>.C
	\end{verbatim}
	seguito da
	\begin{verbatim}
		funzione_da_eseguire(eventuali argomenti)
	\end{verbatim}
	\item per uscire da ROOT digitare il comando \texttt{.q}
\end{itemize}
