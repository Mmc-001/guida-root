Per avviare ROOT da terminale non è sufficiente scrivere semplicemente \texttt{root} e premere INVIO, perchè va "pescato" il suo funzionamento dalla cartella \texttt{root} creata in fase di installazione. I seguenti passaggi permettono di modificare i file di configurazione del Terminale, in modo da non dover rieseguire la stessa stringa di codice ogni volta che si riapre il Terminale prima di eseguire ROOT.
\begin{itemize}
	\item Aprire col Finder la cartella utente, selezionare col tasto destro la cartella \texttt{root} creata in fase di installazione e cliccare \texttt{Ottieni informazioni}: alla dicitura "Situato in" compare il percorso della cartella, copiarlo cliccandoci su col tasto destro (\texttt{Copia come percorso})
	\item Aprire il Terminale verificare il tipo di shell utilizzato, con il comando
	\begin{verbatim}
		echo $0
	\end{verbatim}
	Lo shell di default è \texttt{zsh}, ma è possibile utilizzare anche \texttt{bash} in modo equivalente (a \hyperref{https://www.howtogeek.com/444596/how-to-change-the-default-shell-to-bash-in-macos-catalina/}{}{}{questo link} è spiegata la procedura per cambiare tra uno e l'altro: ai fini dell'utilizzo di ROOT \texttt{zsh} funziona, ed è pertanto sconsigliato il cambio di shell agli utenti meno esperti)
	\item Se lo shell in uso è \texttt{zsh}, digitare nel terminale il comando
	\begin{verbatim}
		sudo vim .zshrc
	\end{verbatim}
	in questo modo si apre (o si crea se non esiste) con i privilegi da amministratore (\texttt{sudo}) il file di configurazione nascosto \texttt{.zshrc} conl'editor di testo Vim
	\item Se lo shell in uso è \texttt{bash}, in modo analogo digitare nel terminale il comando
	\begin{verbatim}
		sudo vim .bashrc
	\end{verbatim}
	oppure
	\begin{verbatim}
		sudo vim .bash_profile
	\end{verbatim}
	\item Una volta aperto con Vim il file di configurazione nascosto con la procedura appena descritta, cliccare I per portarsi nella modalità inserimento di Vim (compare la scritta INSERT al fondo dello schermo), poi portarsi con le frecce al fondo del file e incollare la stringa
	\begin{verbatim}
		source <percorso_root>/root/bin/thisroot.sh
	\end{verbatim}
	in cui al posto di \texttt{<percorso\_root>} va incollato il percorso della cartella \texttt{root}, trovato al primo passaggio di questa lista
	\item Uscire dalla modalità inserimento di Vim cliccando ESC, poi salvare il file digitando il comando \texttt{:wq} e premendo INVIO
	\newpage
	\item Se tutto ha funzionato correttamente, dopo aver chiuso e riaperto il terminale se si digita il comando \texttt{root} dovrebbe avviarsi ROOT con un output simile a
	\begin{verbatim}
		 ------------------------------------------------------------------
		| Welcome to ROOT 6.28/06                        https://root.cern |
		| (c) 1995-2023, The ROOT Team; conception: R. Brun, F. Rademakers |
		| Built for macosx64 on Nov 25 2023, 11:48:00                      |
		| From heads/latest-stable@7745d36d                                |
		| With Apple clang version 15.0.0 (clang-1500.0.40.1)              |
		| Try '.help'/'.?', '.demo', '.license', '.credits', '.quit'/'.q'  |
		 ------------------------------------------------------------------
		
		root [0]
	\end{verbatim}
\end{itemize}