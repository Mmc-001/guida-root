Questa breve guida ha lo scopo di illustrare passo passo la procedura di installazione e setup del software ROOT in ambiente macOS. Si descrivono tre procedure alternative:
\begin{itemize}
    \item Una di livello base, che sfrutta l'utilizzo del packagemanager \texttt{Homebrew} per installare una distribuzione pre-compilata di ROOT (sezione \ref{}).
    \item Una di livello medio, che prevede l'utilizzo del terminale per installare una distribuzione pre-compilata di ROOT con qualche accortezza in più (sezione \ref{}).
    \item Una di livello avanzato, che illustra la procedura di build locale di ROOT da codice sorgente, scaricato tramite clonazione della repository Github ufficiale.
\end{itemize}
Si illustra anche qualche nozione basilare sull'utilizzo di comandi da shell e dell'editor di testo Vim.